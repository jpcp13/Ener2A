\documentclass{exam}

\usepackage[utf8]{inputenc}
\usepackage{amsthm}
\usepackage{amssymb}
\usepackage{amsfonts}
\usepackage[francais]{babel}
\usepackage{listings}

\title{TP1\\ Résolution numérique de $f(x) = 0$}

\begin{document}
\maketitle

Tous les fichiers créés dans ce TP devront être rangés dans un dossier nommé TP1.
On écrira le code dans un fichier nommé \texttt{tp1.c}.\\

\begin{questions}

\question
Création d'une fonction simple et écriture de ses valeurs dans un fichier texte.
\begin{parts}
\part
Créer une fonction \texttt{f} prenant en argument $x$, de type \texttt{double}, et renvoyant  $x^3 - 3x + 1$. 
\part
Créer une fonction \texttt{write\_values} prenant en arguments $a, b, h$ , de type \texttt{double}, et qui écrit dans un fichier nommé \texttt{values.txt} tous les couples $x, f(x)$ (un couple par ligne), pour $x$ variant sur $[a, b]$ avec un pas de $h$.
\part
Ecrire la fonction main sous la forme
\begin{lstlisting}[basicstyle=\small]
int main(int argc, char *argv[]){
	double a, b, h;
	sscanf(argv[1], "%lf", &a);
	sscanf(argv[2], "%lf", &b);
	sscanf(argv[3], "%lf", &h);
	write_values(a, b, h);
	return EXIT_SUCCESS;
}
\end{lstlisting}
\part
En ligne de commande,\\
compiler \texttt{tp1.c} sous la forme \texttt{gcc tp1.c -o tp1} et vérifier que l'exécutable \texttt{tp1} est créé,\\
exécuter \texttt{./tp1 a b h}, avec $a, b, h$ des valeurs au choix; vérifierla création du fichier \texttt{values.txt}.
\part
Localiser les zéros de la fonction $f$ et en donner des encadrements à la précision $h=0.1$.
\end{parts}

\question
Recherche d'un zéro d'une fonction au moyen de la {\bf méthode du point fixe}; notion python abordée : boucle \texttt{for}.
\begin{parts}
\part
Vérifier algébriquement que les points fixes de $g$ sont les zéros de $f$. Ecrire la fonction $g(x) = 1 + 1/x$ sous forme d'une fonction python.
\part
Calculer les $25$ premiers termes de la suite définie par $x_{n+1} = g(x_n)$ et $x_0 = 1.0$ et enregistrer ces termes dans une liste.
\part
Observer la convergence de cette suite vers l'un des deux zéros de $f$.
\part
Dessiner, sur papier en repère orthonormé, la fonction $g$ ainsi que la première bisectrice. Expliquer à partir du graphique la convergence observée.
\end{parts}

\question
Implémentation d'une fonction python \texttt{point\_fixe}. Notions python abordées : boucle \texttt{while}; aspect fonctionnel de python, c'est-à-dire qu'une fonction peut prendre en argument une fonction. Notion mathématique abordée : point fixe attractif, point fixe répulsif.
\begin{parts}
\part
Ecrire une fonction python \texttt{point\_fixe} qui prend en arguments une fonction $g$, une valeur initiale $x_0$, un réel positif $\epsilon$, et qui renvoie une approximation $r$ d'un point fixe de la fonction $g$. Condition d'arrêt : $\vert x_{n+1} - x_n \vert < \epsilon$
\part Tests
\begin{subparts}
\subpart
Tester \texttt{point\_fixe} avec $g(x) = 1 + 1/x, x_0 = 1.6, \epsilon = 10^{-12}$. Vers quelle racine y-a-t'il convergence ?
\subpart
Tester \texttt{point\_fixe} avec $g(x) = 1 + 1/x, x_0 = -0.6, \epsilon = 10^{-12}$. Vers quelle racine y-a-t'il convergence ?
\subpart 
Sur le graphique, examiner la pente de $g$ aux points fixes. Un point fixe $r$ d'une fonction $g$ tel que $\vert g'(r) \vert < 1$ est dit attractif; si $\vert g'(r) \vert > 1$ il est dit répulsif. A partir de ça, expliquer le résultat des tests ci-dessus.
\end{subparts}
\end{parts}


\question
On veut, ici encore, calculer un zéro d'une fonction $f$. La {\bf méthode de Newton} consiste à appliquer la méthode du point fixe à la fonction 
$g(x) = x - \frac{f(x)}{f'(x)}$.
\begin{parts}
\part
Vérifier algébriquement que les points fixes de $g$ sont les zéros de $f$.
\part
Interpréter géométriquement la méthode de Newton.
\part
Ecrire une fonction python \texttt{newton} qui prend en arguments une fonction $f$, sa dérivée $df$, une valeur initiale $x_0$, un réel positif $\epsilon$, et qui renvoie une approximation $r$ d'une racine de la fonction $f$. Condition d'arrêt : $\vert x_{n+1} - x_n \vert < \epsilon$.
\part Tests
\begin{subparts}
\subpart
Tester \texttt{newton} avec $f(x) = x^2 - x - 1, x_0 = 1.0, \epsilon = 10^{-12}$
\subpart
Tester \texttt{newton} avec $f(x) = x^2 - x - 1, x_0 = -1.0, \epsilon = 10^{-12}$
\end{subparts}
\end{parts}

\question
Toujours à la recherche des zéros de $f$, la {\bf méthode de la sécante} est une méthode itérative où chaque approximation est construite à partir des deux approximations précédentes. On doit donc partir de deux valeurs initiales distinctes, $x_0, x_1$ (en général les bornes d'un encadrement de la racine cherchée), puis on calcule par récurrence les termes de la suite $x_{n+1} = x_n - \frac{x_n - x_{n-1}}{f(x_n) - f(x_{n-1})}f(x_n)$. L'avantage sur la méthode de Newton est qu'on n'a pas besoin de la dérivée de $f$.
\begin{parts}
\part
Vérifier algébriquement que les points fixes de $g$ sont les zéros de $f$.
\part
Interpréter géométriquement la méthode de la sécante.
\part
Ecrire une fonction python \texttt{secante} qui prend en arguments une fonction $f$, deux valeurs initiales $x_0, x_1$, un réel positif $\epsilon$, et qui renvoie une approximation $r$ d'une racine de la fonction $f$. Condition d'arrêt : $\vert x_{n+1} - x_n \vert < \epsilon$.
\part Tests
\begin{subparts}
\subpart
Tester \texttt{secante} avec $f(x) = x^2 - x - 1, x_0 = 1.5, x_1 = 2.0, \epsilon = 10^{-12}$
\subpart
Tester \texttt{secante} avec $f(x) = x^2 - x - 1, x_0 = -1.0, x_1 = -0.5, \epsilon = 10^{-12}$
\end{subparts}
\end{parts}

\question
La {\bf méthode de dichotomie} suppose que $f$ est continue sur un intervalle $(a, b)$ et change de signe sur cet intervalle; on est donc assuré que $f$ possède un zéro sur cet intervalle. Ensuite on coupe $(a, b)$ en deux et on garde celui des deux intervalles où $f$ change de signe. On obtient donc un nouvel encadrement $a, b$ deux fois plus petit. On répète l'opération jusqu'à obtenir la précision souhaitée.
\begin{parts}
\part
Ecrire une fonction python \texttt{dichotomie} qui prend en arguments une fonction $f$, deux valeurs initiales $a, b$, un réel positif $\epsilon$, et qui renvoie un encadrement $a, b$ d'une racine de la fonction $f$. Condition d'arrêt : $\vert b - a \vert < \epsilon$.
\part Tests
\begin{subparts}
\subpart
Tester \texttt{dichotomie} avec $f(x) = x^2 - x - 1, a = 1.5, b = 2.0, \epsilon = 10^{-12}$
\subpart
Tester \texttt{dichotomie} avec $f(x) = x^2 - x - 1, a = -1.0, b = 0.0, \epsilon = 10^{-12}$
\end{subparts}
\end{parts}

\question
Examinons maintenant la vitesse de convergence de ces méthodes. 
\begin{parts}
\part
Dans chacune des méthodes, incorporer un compteur qui compte le nombre d'itérations \texttt{nbiter} effectuées et placer \texttt{nbiter} dans le return de la fonction. Par exemple, le return de la fonction \texttt{newton} s'écrira \texttt{return r, nbiter}. Lorsqu'on appelera \texttt{newton}, on écrira donc \texttt{r, nbiter = newton(f, df, x0, epsi)}
\part
Pour chacune des méthodes, faire varier la valeur de $\epsilon$ et compter le nombre d'itérations effectuées. Reporter les résultats dans un tableau, par exemple :\\
\begin{center}
\begin{tabular}{l | c c c c c}
{} & $10^{-3}$ & $10^{-6}$ & $10^{-9}$ & $10^{-12}$ & $10^{-15}$ \\
\hline
\texttt{point\_fixe} & {} & {} & {} & {} & {} \\
\texttt{newton} & {} & {} & {} & {} & {} \\
\texttt{secante} & {} & {} & {} & {} & {} \\
\texttt{dichotomie} & {} & {} & {} & {} & {} 
\end{tabular}
\end{center}
\end{parts}

\question
Rédiger le compte-rendu du TP1, un compte-rendu par binôme, dans l'un des formats (que l'on pourra combiner, si besoin) :
\begin{itemize}
\item
papier
\item
ipython notebook (extension .ipynb)
\item
latex (extension .tex)
\item
libreoffice (extension .odt)
\end{itemize}

\end{questions}

\end{document}



