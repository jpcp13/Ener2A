\documentclass{exam}

\usepackage[utf8]{inputenc}
\usepackage{amsmath}
\usepackage{amsthm}
\usepackage{amssymb}
\usepackage{amsfonts}
\usepackage[francais]{babel}
\usepackage{listings}

\title{TP1 : Résolution numérique de $f(x) = 0$}

\begin{document}
\maketitle

\begin{questions}

\question
Programme \texttt{values.c} : création d'une fonction simple et écriture de ses valeurs dans un fichier texte.
\begin{parts}
\part
dessiner sur papier la fonction $f(x) = x^3 - 3x + 1$ et localiser approximativement ses zéros.
\part
Dans \texttt{values.c}, écrire une fonction \texttt{f} prenant en argument $x$, de type \texttt{double}, et renvoyant  $x^3 - 3x + 1$. 
\part
Ecrire une fonction \texttt{write\_values} prenant en arguments \texttt{a, b, h}, de type \texttt{double}, et qui écrit dans un fichier nommé \texttt{values.txt} les couples $x, f(x)$ - un couple par ligne -, pour $x$ variant sur $[a, b]$ avec un pas de $h$.
\part
Ecrire une fonction \texttt{main} prenant en arguments \texttt{a, b, h} (ces arguments seront passés en ligne de commande au moment de l'exécution du programme); la fonction \texttt{main} fera ensuite appel à \texttt{write\_values} avec les arguments \texttt{a, b, h}.
\part
En ligne de commande, compiler \texttt{values.c} sous la forme \texttt{gcc values.c -o values}, vérifier que le fichier exécutable \texttt{values} est créé, puis exécuter \texttt{./values a b h}, avec $a, b, h$ des valeurs au choix; vérifier la création du fichier \texttt{values.txt}.
\part
Examiner le contenu du fichier \texttt{values.txt} et donner des encadrements des zéros de $f$ à la précision $h=0.1$.
\end{parts}


\question
Programme \texttt{dichotomie.c} : Soit $f$ une fonction continue sur un intervalle $(a, b)$; la {\bf méthode de dichotomie} suppose que $f$ change de signe sur l'intervalle; on est donc assuré que $f$ possède un zéro sur cet intervalle. Ensuite on coupe l'intervalle $(a, b)$ en deux et on garde celui des deux intervalles où $f$ change de signe. On obtient donc un nouvel encadrement $(a, b)$ deux fois plus petit. On répète l'opération jusqu'à obtenir la précision souhaitée.
\begin{parts}
\part
Illustrer quelques étapes de la construction précédente sur un dessin avec $f(x) = x^3 - 3x + 1$.
\part
\begin{subparts}
\subpart
Dans \texttt{dichotomie.c}, écrire la fonction \texttt{f} ci-dessus.
\subpart
Ecrire une fonction \texttt{dichotomie} qui prend en arguments \texttt{a, b, epsilon} de type \texttt{double} et qui écrit dans un fichier \texttt{dichotomie.txt} les encadrements successifs $(a, b)$ - un couple par ligne; condition d'arrêt : $\lvert b - a \rvert < \epsilon$.
\end{subparts}
\part
Compiler \texttt{gcc dichotomie.c -o dichotomie}, exécuter \texttt{./dichotomie a b epsilon} pour différentes valeurs de \texttt{a b epsilon} et vérifier vos résultats dans le fichier \texttt{dichotomie.txt}.
\end{parts}


\question
Programme \texttt{newton.c} : {\bf méthode de Newton}.
\begin{parts}
\part
Soit $f$ une fonction quelconque et $x_0$ un réel, appelé valeur initiale; soit $T$ la tangente à la courbe représentative de $f$, au point $(x_0, f(x_0))$; $T$ coupe l'axe des $x$ au point d'abscisse $x_1$; exprimer $x_1$ à l'aide des quantités $x_0, f(x_0), f'(x_0)$.
\part
Illustrer la construction ci-dessus sur un dessin avec $f(x) = x^3 - 3x + 1$ et $x_0 = 2.0$; que vaut $x_1$ ?
\part
La méthode de Newton consiste à prendre $x_1$ comme nouvelle valeur initiale, et à répéter le processus ci-dessus; on obtient ainsi des valeurs $x_2, x_3, \cdots$, etc; sur le dessin précédent, vers quoi semble converger la suite $x_n$ ?
\part
Dans \texttt{newton.c}, écrire la fonction \texttt{f} habituelle, puis écrire la fonction \texttt{df}, dérivée de $f$.
\part
Ecrire une fonction \texttt{newton} qui prend en arguments  \texttt{x0, epsilon}, de type \texttt{double} et qui écrit dans un fichier nommé \texttt{newton.txt} les couples $x_n, \lvert x_n - x_{n-1} \rvert$ - un couple par ligne - (la quantité $\lvert x_n - x_{n-1} \rvert$ s'appelle erreur de la méthode de Newton à l'étape $n$). Condition d'arrêt : $\lvert x_n - x_{n-1} \rvert < \epsilon$; la première ligne, correspondant à $n = 0$, ne contiendra que la valeur $x_0$.
\part
Compiler \texttt{gcc newton.c -o newton}, exécuter \texttt{./newton x0 epsilon} pour différentes valeurs de \texttt{x0 epsilon} et vérifier vos résultats dans le fichier \texttt{newton.txt}.
\end{parts}

\question
Programme \texttt{newton\_numdiff.c} : méthode de Newton avec {\bf dérivée numérique}.\\
Pour éviter le calcul à la main de la dérivée de $f$, qui peut s'avérer difficile, voire impossible, on peut utiliser l'approximation mathématique 
\begin{equation}
\label{numdiff}
f'(x) \approx \frac{f(x+h) - f(x-h)}{2h}
\end{equation}
 où $h$ est un réel positif fixé.
\begin{parts}
\part
Sauvegarder le programme \texttt{newton.c} dans un fichier \texttt{newton\_numdiff.c}
\part
Dans \texttt{newton\_numdiff.c}, supprimer la fonction \texttt{df} et ajouter la constante \texttt{h = 1.0E-8}.
\part
Modifier la fonction \texttt{newton} en accord avec l'approximation (\ref{numdiff}); les résultats du calcul seront écrits dans un fichier \texttt{newton\_numdiff.txt}.

\part
Compiler et exécuter \texttt{newton\_numdiff} et comparer les résultats avec ceux de \texttt{newton}.
\end{parts}

\question
Programme \texttt{secante.c} : {\bf méthode de la sécante}.
\begin{parts}
\part
Soit $f$ une fonction et $x_0, x_1$ deux réels distincts - les valeurs initiales; soit $T$ la droite - appelée sécante - s'appuyant sur les points $(x_0, f(x_0))$ et $(x_1, f(x_1))$; $T$ coupe l'axe des $x$ au point d'abscisse $x_2$; exprimer $x_2$ à l'aide des quantités $x_0, f(x_0), x_1, f(x_1)$.
\part
Illustrer cette construction avec $f(x) = x^3 - 3x + 1$ et $x_0 = 1.0, x_1 = 2.0$; que vaut $x_2$ ?
\part
La méthode de la sécante consiste à prendre $x_1, x_2$ comme nouvelles valeurs initiales, et à répéter le processus; on obtient ainsi des valeurs $x_2, x_3, \cdots$, etc; sur le dessin précédent, vers quoi semble converger la suite $x_n$ ?
\part
Dans \texttt{secante.c}, outre la fonction \texttt{f}, écrire une fonction \texttt{secante} qui prend en arguments  \texttt{x0, x1, epsilon} et qui écrit dans un fichier \texttt{secante.txt} les couples $x_n, \lvert x_n - x_{n-1} \rvert$. Condition d'arrêt : $\lvert x_n - x_{n-1} \rvert < \epsilon$.
\part
Compiler \texttt{gcc secante.c -o secante}, exécuter \texttt{./secante x0 x1 epsilon} pour différentes valeurs \texttt{x0 x1 epsilon} et vérifier vos résultats dans \texttt{secante.txt}.
\end{parts}


\question
\begin{parts}
\part
Choisir un zéro de $f$; pour chacune des méthodes présentées, fixer des valeurs initiales, faire varier la valeur de $\epsilon$ et compter le nombre d'itérations effectuées; reporter les résultats dans le tableau :
\begin{center}
\begin{tabular}{r | c c c c c}
{$\epsilon$} & $10^{-3}$ & $10^{-6}$ & $10^{-9}$ & $10^{-12}$ & $10^{-15}$ \\
\hline
\texttt{dichotomie} & {} & {} & {} & {} & {} \\
\texttt{newton} & {} & {} & {} & {} & {} \\
\texttt{sécante} & {} & {} & {} & {} & {} 

\end{tabular}
\end{center}
\part
Rédiger un court compte-rendu, contenant explications, commentaires, dessins, table.
\end{parts}



\question
Programme \texttt{newton2.c} : {\bf méthode de Newton, multi-dimensionnelle}.\\
Dans cette partie nous montrons comment généraliser la méthode de Newton pour calculer un zéro d'un sytème de plusieurs équations à plusieurs inconnues.
\begin{parts}
 
\part
Dans cet exemple, on considère un système de deux équations à deux inconnues $x_0, x_1$; on posera \texttt{X} = $(x_0, x_1)$.
Dans \texttt{newton2.c}, écrire la fonction \texttt{f} définie par $f(x_0, x_1) = (x_0^2 + 2x_1^2 - 1.0, x_0x_1)$; la fonction \texttt{f} prendra en argument \texttt{X}, et retournera \texttt{fX}; \texttt{X} et \texttt{fX} sont des \texttt{arrays} de longueur $2$ et de type \texttt{double}.

\part
Ecrire la fonction \texttt{df}, dérivée de $f$; mathématiquement, la dérivée de $f$ au point $(x_0, x_1)$ est la matrice, dite Jacobienne :
\begin{equation}
df(x) = \begin{pmatrix}
\dfrac{\partial f_0}{\partial x_0} & \dfrac{\partial f_0}{\partial x_1} \\
\dfrac{\partial f_1}{\partial x_0} & \dfrac{\partial f_1}{\partial x_1} 
\end{pmatrix}
\end{equation}
avec $f_0 = x_0^2 + 2x_1^2 - 1.0, f_1 = x_0x_1$; la fonction \texttt{df} prendra en argument \texttt{X} et retournera \texttt{dfX}, avec \texttt{dfX} = $[\frac{\partial f_0}{\partial x_0}, \frac{\partial f_0}{\partial x_1}, \frac{\partial f_1}{\partial x_0}, \frac{\partial f_1}{\partial x_1}]$; \texttt{X} et \texttt{dfX} sont des \texttt{arrays} de type \texttt{double}, l'un de longueur $2$, l'autre de longueur $4$.

\part
La méthode de Newton consiste à choisir une valeur initiale $X_0 \in \mathbb{R}^2$ et à calculer la suite définie par la récurrence $X_{n} = X_{n-1} - df(X_{n-1})^{-1}X_{n-1}$. Pour de nombreux choix de $X_0$, la méthode de Newton converge, très rapidement, vers un zéro de $f$; il peut arriver cependant, pour certains choix de $X_0$, que la méthode ne converge pas.\\

\begin{subparts}
\subpart
Ecrire une fonction \texttt{inverse}, qui prend une matrice \texttt{mat} carrée de taille $2$, \texttt{array} de type \texttt{double} et de longueur $4$ et qui renvoie son inverse \texttt{imat}.
\subpart
Ecrire une fonction \texttt{multiply}, qui prend une matrice \texttt{mat}, ainsi qu'un vecteur \texttt{v} et qui renvoie le produit de \texttt{mat} par \texttt{v}; on notera \texttt{w} ce produit; \texttt{v} et \texttt{w} sont des \texttt{arrays} de type \texttt{double} et de longueur $2$.
\end{subparts}

\part
Ecrire une fonction \texttt{newton2} qui prend en arguments une valeur initiale \texttt{X0} et un scalaire \texttt{epsilon};
\texttt{newton2} écrira dans un fichier nommé \texttt{newton2.txt} les couples $X_n$ ainsi que l'erreur $\lVert X_n - X_{n-1} \rVert$ - il y aura donc valeurs numériques par ligne, et seulement deux valeurs sur la première ligne; en pratique, pour l'erreur $\lVert X_n - X_{n-1} \rVert$, on pourra prendre au choix\\
$\sqrt{(X_n[0] - X_{n-1}[0])^2 + (X_n[1] - X_{n-1}[1])^2}$\\
 ou $\lVert X_n - X_{n-1} \rVert = \lvert X_n[0] - X_{n-1}[0] \rvert + \lvert X_n[1] - X_{n-1}[1] \rvert$, si la fonction racine carrée n'est pas disponible.
\part
Compiler \texttt{gcc newton2.c -o newton2}, exécuter \texttt{./newton2 X0[0] X0[1] epsilon} pour différentes valeurs de \texttt{X0 epsilon} et vérifier vos résultats dans le fichier \texttt{newton2.txt}.
\end{parts}



\end{questions}
\end{document}



