\documentclass{exam}

\usepackage[utf8]{inputenc}
\usepackage{amsmath}
\usepackage{amsthm}
\usepackage{amssymb}
\usepackage{amsfonts}
\usepackage[francais]{babel}
\usepackage{listings}

\title{TP1\\ Résolution numérique de $f(x) = 0$}

\begin{document}
\maketitle

Tous les fichiers créés dans ce TP devront être rangés dans un dossier nommé TP1.
On écrira le code dans un fichier nommé \texttt{tp1.c}.\\

\begin{questions}

\question
Programme \texttt{values.c} : création d'une fonction simple et écriture de ses valeurs dans un fichier texte.
\begin{parts}
\part
dessiner sur papier la fonction $f(x) = x^3 - 3x + 1$ et localiser approximativement ses zéros.
\part
Dans \texttt{values.c}, écrire une fonction \texttt{f} prenant en argument $x$, de type \texttt{double}, et renvoyant  $x^3 - 3x + 1$. 
\part
Ecrire une fonction \texttt{write\_values} prenant en arguments $a, b, h$ , de type \texttt{double}, et qui écrit dans un fichier nommé \texttt{values.txt} tous les couples $x, f(x)$ (un couple par ligne), pour $x$ variant sur $[a, b]$ avec un pas de $h$.
\part
Ecrire la fonction \texttt{main} sous la forme
\begin{lstlisting}[basicstyle=\small]
int main(int argc, char *argv[]){
	double a, b, h;
	sscanf(argv[1], "%lf", &a);
	sscanf(argv[2], "%lf", &b);
	sscanf(argv[3], "%lf", &h);
	write_values(a, b, h);
	return EXIT_SUCCESS;
}
\end{lstlisting}
On dira que \texttt{main} prend en arguments $a, b, h$.
\part
En ligne de commande, compiler \texttt{values.c} sous la forme \texttt{gcc values.c -o values} et vérifier que l'exécutable \texttt{values} est créé, puis exécuter \texttt{./values a b h}, avec $a, b, h$ des valeurs au choix; vérifier la création du fichier \texttt{values.txt}.
\part
A l'aide du fichier \texttt{values.txt}, donner des encadrements des zéros de $f$ à la précision $h=0.01$.
\end{parts}


\question
Programme \texttt{newton.c} : implémentation de la {\bf méthode de Newton}.
%$g(x) = x - \frac{f(x)}{f'(x)}$.
\begin{parts}
\part
Soit $f$ une fonction quelconque et $x_0$ un réel, appelé valeur initiale; soit $T$ la tangente à la courbe représentative de $f$, au point $(x_0, f(x_0))$; $T$ coupe l'axe des $x$ au point d'abscisse $x_1$; exprimer $x_1$ à l'aide des quantités $x_0, f(x_0), f'(x_0)$.
\part
Illustrer la construction ci-dessus sur un dessin avec $f(x) = x^3 - 3x + 1$ et $x_0 = 2.0$; que vaut $x_1$ ?
\part
La méthode de Newton consiste à prendre $x_1$ comme nouvelle valeur initiale, et à répéter le processus ci-dessus; on obtient ainsi des valeurs $x_2, x_3, \cdots$, etc; sur le dessin précédent, vers quoi semble converger la suite $x_n$ ?

\part
Dans \texttt{newton.c}, écrire une fonction \texttt{f} prenant en argument $x$, de type \texttt{double} et renvoyant  $x^3 - 3x + 1$, puis écrire une fonction \texttt{df} représentant la dérivée de $f$.

\part
Ecrire une fonction \texttt{newton} qui prend en arguments  $x_0, \epsilon$, de type \texttt{double} et qui écrit dans un fichier nommé \texttt{newton.txt} les couples $x_n, \lvert x_n - x_{n-1} \rvert$ (un couple par ligne; la quantité $\lvert x_n - x_{n-1} \rvert$ s'appelle erreur de la méthode de Newton à l'étape $n$); condition d'arrêt : $\lvert x_n - x_{n-1} \rvert < \epsilon$; la première ligne, correspondant à $n = 0$, ne contiendra que la valeur $x_0$.
\end{parts}

\question
Toujours à la recherche des zéros de $f$, la {\bf méthode de la sécante} est une méthode itérative où chaque approximation est construite à partir des deux approximations précédentes. On doit donc partir de deux valeurs initiales distinctes, $x_0, x_1$ (en général les bornes d'un encadrement de la racine cherchée), puis on calcule par récurrence les termes de la suite $x_{n+1} = x_n - \frac{x_n - x_{n-1}}{f(x_n) - f(x_{n-1})}f(x_n)$. L'avantage sur la méthode de Newton est qu'on n'a pas besoin de la dérivée de $f$.
\begin{parts}
\part
Vérifier algébriquement que les points fixes de $g$ sont les zéros de $f$.
\part
Interpréter géométriquement la méthode de la sécante.
\part
Ecrire une fonction python \texttt{secante} qui prend en arguments une fonction $f$, deux valeurs initiales $x_0, x_1$, un réel positif $\epsilon$, et qui renvoie une approximation $r$ d'une racine de la fonction $f$. Condition d'arrêt : $\vert x_{n+1} - x_n \vert < \epsilon$.
\part Tests
\begin{subparts}
\subpart
Tester \texttt{secante} avec $f(x) = x^2 - x - 1, x_0 = 1.5, x_1 = 2.0, \epsilon = 10^{-12}$
\subpart
Tester \texttt{secante} avec $f(x) = x^2 - x - 1, x_0 = -1.0, x_1 = -0.5, \epsilon = 10^{-12}$
\end{subparts}
\end{parts}

\question
La {\bf méthode de dichotomie} suppose que $f$ est continue sur un intervalle $(a, b)$ et change de signe sur cet intervalle; on est donc assuré que $f$ possède un zéro sur cet intervalle. Ensuite on coupe $(a, b)$ en deux et on garde celui des deux intervalles où $f$ change de signe. On obtient donc un nouvel encadrement $a, b$ deux fois plus petit. On répète l'opération jusqu'à obtenir la précision souhaitée.
\begin{parts}
\part
Ecrire une fonction python \texttt{dichotomie} qui prend en arguments une fonction $f$, deux valeurs initiales $a, b$, un réel positif $\epsilon$, et qui renvoie un encadrement $a, b$ d'une racine de la fonction $f$. Condition d'arrêt : $\vert b - a \vert < \epsilon$.
\part Tests
\begin{subparts}
\subpart
Tester \texttt{dichotomie} avec $f(x) = x^2 - x - 1, a = 1.5, b = 2.0, \epsilon = 10^{-12}$
\subpart
Tester \texttt{dichotomie} avec $f(x) = x^2 - x - 1, a = -1.0, b = 0.0, \epsilon = 10^{-12}$
\end{subparts}
\end{parts}

\question
Examinons maintenant la vitesse de convergence de ces méthodes. 
\begin{parts}
\part
Dans chacune des méthodes, incorporer un compteur qui compte le nombre d'itérations \texttt{nbiter} effectuées et placer \texttt{nbiter} dans le return de la fonction. Par exemple, le return de la fonction \texttt{newton} s'écrira \texttt{return r, nbiter}. Lorsqu'on appelera \texttt{newton}, on écrira donc \texttt{r, nbiter = newton(f, df, x0, epsi)}
\part
Pour chacune des méthodes, faire varier la valeur de $\epsilon$ et compter le nombre d'itérations effectuées. Reporter les résultats dans un tableau, par exemple :\\
\begin{center}
\begin{tabular}{l | c c c c c}
{} & $10^{-3}$ & $10^{-6}$ & $10^{-9}$ & $10^{-12}$ & $10^{-15}$ \\
\hline
\texttt{point\_fixe} & {} & {} & {} & {} & {} \\
\texttt{newton} & {} & {} & {} & {} & {} \\
\texttt{secante} & {} & {} & {} & {} & {} \\
\texttt{dichotomie} & {} & {} & {} & {} & {} 
\end{tabular}
\end{center}
\end{parts}

\question
Rédiger le compte-rendu du TP1, un compte-rendu par binôme, dans l'un des formats (que l'on pourra combiner, si besoin) :
\begin{itemize}
\item
papier
\item
ipython notebook (extension .ipynb)
\item
latex (extension .tex)
\item
libreoffice (extension .odt)
\end{itemize}

\end{questions}

\end{document}



